

%%
%% The following code sets up the document formatting
%%

%this assumes that res_yy.sty is in some path
\documentstyle[hyperref, margin, line]{res_yy}

\hypersetup{backref,pdfpagemode=Full,colorlinks=true,backref}

\addtolength{\oddsidemargin}{-0.45in}
\addtolength{\voffset}{-0.55in}
\addtolength{\textwidth}{1.00in} \addtolength{\textheight}{1.50in}

\renewcommand{\namefont}{\LARGE\emph}



%%
%% The following code defines some macros for terms which have raised font
%% (ie 4\fourth would result 4th with the 'th' raised (superscripted)
%%

\def\Cplusplus{{\rm C\raise.5ex\hbox{\small ++}}}
\def\CSharp{{\rm C\raise.5ex\hbox{\small \#}}}
% 'st' 'nd' 'rd' 'th' superscripts for numbers
\def\first{{\raise.5ex\hbox{\small st}}}
\def\second{{\raise.5ex\hbox{\small nd}}}
\def\third{{\raise.5ex\hbox{\small rd}}}
\def\fourth{{\raise.5ex\hbox{\small th}}}



%%
%% starting the actual document
%%

\begin{document}

%the name in big fonts at the top of resume
%this is left aligned
\name{Oren Mazor}

%this is right aligned
\address{Toronto, Ontario, Canada}
\address{(613) 899-5788 \ttfamily $<$oren.mazor at protonmail com$>$}

\begin{resume}

%%
%% This section of code is inelegant, but I'm too lazy to fix it
%%

\section{\textsc{Summary}}
A radically candid, creative software engineer with a strong cross-technology skill-set.\\\\Looking for opportunities to leverage my existing skills in Site-Reliability Engineering, Performant design, and related DevOps work.\\\\My goal is to ship fantastic products that make people's lives better.

\section{\textsc{Education}}

\textit{Bachelor of Computer Science}, Class of 2006. \\
Concentration in Information Security \\
Minor in Business \\
Carleton University, Ottawa, Canada \\
Thesis: \textit{''Practical Mitigation of Malicious Network Resource Consumers''} \\
Supervised by Dr. Anil Somayaji. 

%%
%% the meat of the resume starts now
%%

\begin{formatb}
  \employer{l}\title{r}\\
  \location{l}\dates{r}\\
  \body\\
\end{formatb}

\section{\textsc{Research Topics}}

Observabilty, Incident Management, Network Security, Homelabs, Data Privacy, Capacity Planning.

\section{\textsc{Online}}
https://www.github.com/orenmazor\\
https://www.linkedin.com/in/orenmazor/

\section{\textsc{Qualifications}}

\emph{Preferred Programming Languages}: Golang, Python.

\emph{Scripting:} Python, Perl, Ruby, Lua.

\emph{Databases}: PostgreSQL, MySQL, Redshift/HADOOP.

\emph{Platforms}: Hashicorp stack (Terraform/Nomad), AWS (significant expertise), GCP, Digital Ocean.

\section{\textsc{Primary Professional Experience}}

\employer{\textbf{1Password}}

\title{Staff Production Engineer}
        \location{Toronto, Canada}
\dates{2022 - Present}
\begin{position}
\begin{itemize}
\item Led a team of production engineers in modernizing our observability infrastructure from disjoint tooling and into a uniform posture on Datadog. We encouraged the usage of scorecards, service catalog, and a variety of effort multiplying middleware to encourage tracing use.
\item Published and prototyped RFD for a new Service-Oriented-Architecture platform built on top of K8s. This was a collaborative cross-engineering process, where I made sure to find a satisfying compromise for all teams. The end result is a product that can be easily onboarded by a team and provided automated CICD as well as observability with little to no need for K8s expertise. It was recently used as a CD target by all projects of a hackathon, as well as the basis of a linkerd strategy for uniform golden signals for deployed services.
\item Managed and mentored SRE principles, golang engineering principles, and team topologies principles to my team as it matured into a team that provides enablement services to our engineering org.
\item Provided technical leadership in site reliability engineering to a rapidly maturing engineering organization (from 50 to 500 engineers during my tenure).
\end{itemize}
\end{position}

\title{Senior DevOps Engineer}
        \location{Toronto, Canada}
\dates{2019 - 2022}
\begin{position}
\begin{itemize}
\item Prototyped a grey labelling approach of the cloud platform from AWS and into a private data center for our clients, moving our main workloads from virtual machines and into openshift.
\item Built a PII preserving datalake on AWS Athena that ingests from a variety of global data centers using Pandas and AWS Batch. I built two parallel frontrooms, Metabase for engineers, and Looker for GTM teams.
\item Built and staffed out a BI team on top of the datalake to provide business services to users.
\end{itemize}
\end{position}


\employer{\textbf{Vium}}
\title{Senior DevOps Engineer}
        \location{San Mateo, California (Remote)}
\dates{2018 - 2020}
\begin{position}
\begin{itemize}
\item Focus on developer happiness (migrated CICD infrastructure from Jenkins to Gitlab)
\item Supported a variety of computer vision, machine learning, and spark cluster stacks. This included working on individual jobs, maintaining cluster provisioning, and introducing a variety of performance analytics tools.
\item Created a tool for building customizeable AWS stacks using Troposphere and CloudFormation, including databases on RDS, custom Dockerfiles, and a variety of cloudwatch and SQS alarms.
\end{itemize}
\end{position}

\employer{\textbf{ReCharge Payments}}
\title{Devops Lead}
        \location{Santa Monica, California (Remote)}
\dates{2016 - 2018}
\begin{position}
\begin{itemize}
\item Introduced and managed CI/CD processes.
\item Worked on dockerizing the codebase and migrated the infrastructure from an adhoc deploy supervisorctl/runit method to a dockerized cluster managed on Digital Ocean. Prototyped ECS/EKS orchestration alternatives.
\item Implemented SRE principles to monitor security and performance on a variety of services provided to merchants and customers using modern industry standards (statsd, newrelic, pagerduty).
\item Worked on preparing the infrastructure for the throughput required for BFCM. Implemented a variety of strategies for improving throughput and managing complex business logic flows that are externals bound through safe failover mechanisms (e.g. Stripe, Avalara, UPS, Shopify).
\item Implemented a backgrounding mechanism based on Python's RQ, heavily inspired by Sidekiq/Resque and several other AMQP mechanisms. Added exponential and a variety of LRU backoff strategies to accomodate external resource throttles.
\item Built several core product features from prototype to fully functional core tools, used by thousands of merchants and customers.
\end{itemize}
\end{position}

\employer{\textbf{Rarelogic}}
\title{Lead Platform Developer}
\location{Ottawa Canada}
\dates{2015-2016}
\begin{position}
%\begin{itemize}
%\item Applied SRE and CI/CD mechanisms to aid in modernizing a prototype codebase in preparation for wide access by customers.
%\item Converted a complex microservices architecture from sorta-node.js to rails, added test coverage and proper code release mechanisms.
%\item Stabilized and modernized the database architecture from a customer in-memory redis based to one backed by postgresql and cassandra.
%\item Lived with the horror that is MTA bulk emailing IP reputation management.
%\end{itemize}
\end{position}

\employer{\textbf{Shopify}}
\title{Site Reliability Engineer}
\location{Ottawa Canada}
\dates{2012 - 2015}
\begin{position}
%\begin{itemize}
%\item Shipped multiple products from MVP to Production Ready of new features, including Merchant Admin Console BI reporting tools, making use of data cubes, a custom golang lmdb analytics database, and a variety of real-time ETL stacks.
%\item Worked on the SRE team on flash sales and security infrastructure. Focus on preparation for Black-Friday/Cyber Monday flash sale load.
%\item Shipped the first revision of a performant multi-datacenter HTTP router using Lua and Nginx to replace HAproxy
%\item Helped migrate a large Hadoop Data Cluster from MR1 to MR2. Then prototyped and ran a migration for a 1000 job ETL scheduler. Worked on cluster performance, stability, hardware maintenance, and recoverability. 
%\item Set up, migrated, and administered a Tableau High-Availability cluster. 
%\end{itemize}
\end{position}

\employer{\textbf{Wildbit}}
\title{Software Developer}
\location{Ottawa Canada}
\dates{2011 - 2012}
\begin{position}
%\begin{itemize} 
%\item Ran DevOps for the Postmark stack, which included a variety of windows services, centos servers, and a stack of microservices to fulfill all tasks from searching, sending, MTA maintenance, and UI performance.
%\item Created and delivered a variety of features, including a migration from an admin dashboard powered by MongoDB to one powered by ElasticSearch and CouchDB. Part of this project required the creation of a custom MongoDB to ES replication strategy.
%\end{itemize}
\end{position}

\employer{\textbf{OneChip Photonics}}
\title{Software Developer}
\location{Ottawa Canada}
\dates{May 2009 - Present}
\begin{position}
%\begin{itemize}
%\item Duties included advanced understanding of C\# coding standards, problem identification and solution, deployment strategies for libraries and applications (internal and customer facing). Also responsible for training test automation and electrical engineers in basic software design principles.
%\item Responsible for the creation and maintenance of several test automation software for photooptic devices, with a heavy focus on robotic validation of circuitry using computer vision algorithms.
%\item Designed an internally used extensive API for test automation libraries, test stations, GPIB device drivers, and a LINQ based ORM for manipulating test data results.
%\end{itemize}
\end{position}

\employer{\textbf{Macadamian Technologies}}
\title{Software Developer}
\location{Ottawa Canada}
\dates{June 2006 - January 2009}
\begin{position}
%\begin{itemize}
%\item Worked in a team to design and deliver major features of embedded VOIP desktop and software phones, including Emergency dialing/routing, UI re-design, SMS messaging, and designing feature provisioning using SOAP.
%\item Developed plug-ins for a Microsoft Office tool, as well as developing and maintaining a variety of custom Windows installers using the WiX framework.
%\item Took on customer facing project leadership tasks in project administration, status reporting, workload division, and quality assurance communication in a six person globally distributed project.
%\item Advanced personal understanding of low level embedded systems written in C/C++, such as writing applications for the EFI Framework and VxWorks.
%\item Represented Macadamian in professional and recruiting events, as well as co-editing the corporate blog, and developing the competition monitoring, solution submission, and marking web application for the CS Games 2007 Programming competition (written in PHP with an open source AJAX toolkit).
%\item Created and ran internal training discussion groups (Barcamps) on topics ranging from iPhone/Android SDKs to Semantic Web technologies.
%\end{itemize}
\end{position}

%\employer{\textbf{International Datacasting Corporation}}
%\title{Software Developer}
%\location{Ottawa Canada}
%\dates{June 2004 - June 2006}
%\begin{position}
%%\begin{itemize}
%%\item Implemented a customized ''heartbeat'' monitoring tool for a realtime satellite broadcasting headend network using C\# and a variety of win32 libraries.
%%\item Developed an in-house used Wireshark (C++/GTK) visual plugin for debugging of the control protocol, previously done via notepad and a hex dump.
%%\item Responsible for system maintenance of RedHat Linux and Windows enterprise platforms, and aiding in a variety of projects, from a video transmission tool written in VB, a real-time AJAX scheduling tool written in PHP, and in-depth work with a c/perl head-end satellite receiver management web application.
%%\end{itemize}
%\end{position}

%%
%% We use the same formatting for projects as for work experience
%% Shown below is the formatting used previously
%%
%%  \begin{formatb}
%%    \employer{l}\title{r}\\
%%    \location{l}\dates{r}\\
%%    \body\\
%%  \end{formatb}
%%
%% 
%%  Note that \location is now being used for non-location information
%%

% \section{\textsc{Volunteer Work}}
% \begin{formatb}
%   \employer{l}\title{r}\\
%   \location{l}\dates{r}\\
%   \body\\
% \end{formatb}
%
% \employer{\textbf{Carleton University}}
% \title{Carleton University Senate Senator }
% \location{Ottawa Canada}
% \dates{June 30, 2005 - June 30, 2006}
% \begin{position}
% Senator representing the constituency of Science. Responsible for representing the needs of students as well as bringing student issues to the University Senate.
% \end{position}
%  
% \employer{\textbf{Carleton Computer Science Society}}
% \title{IT Administrator}
% \location{Ottawa Canada}
% \dates{June 30, 2005 - June 30, 2006}
% \begin{position}
% Responsible for all technology oriented facets of the organization, including web/db server administration, networking infrastructure, and web cms administration and upgrading. This was done first through a homebuilt CMS tool, and later using Drupal.
% \end{position}
%
% \employer{\textbf{Carleton University}}
% \title{NUG Representative}
% \location{Ottawa Canada}
% \dates{September 30, 2004 - June 30, 2006}
% \begin{position}
% Student representative for constituency of Computer Science.
% \end{position}
%
% \employer{\textbf{Carleton Computer Science Society}}
% \title{Financial Officer}
% \location{Ottawa Canada}
% \dates{June 30, 2004 - June 30, 2005}
% \begin{position}
% Executive responsible for fiscal operations of the society. 
% \end{position}
%
% \employer{\textbf{Carleton Nexus Project}}
% \title{Assistant Server Administrator }
% \location{Ottawa Canada}
% \dates{November, 2003 - June, 2005}
% \begin{position}
% Setup and maintenance of SQL, mail, ldap, and webservers.
% \end{position}

%%
%% Note that we're redefining the formatting
%% We only have one row of information now, instead of two
%%

\begin{formatb}
  \employer{l}\dates{r}\\
  \body\\
\end{formatb}


\end{resume}
\end{document}
